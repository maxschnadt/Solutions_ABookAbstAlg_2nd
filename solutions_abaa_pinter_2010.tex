\documentclass[11pt, b5paper, draft, fleqn]{book}
\usepackage[inner=2.00cm, outer=2.0cm, top=2.0cm, bottom=2.0cm]{geometry}
%
%
\usepackage[english]{babel}
%
%
\usepackage{amsmath}
\usepackage{amsfonts}
%\usepackage{amssymb}
\usepackage{amsthm}

\theoremstyle{remark}
\newtheorem*{solution}{Solution}
\newtheorem*{prop}{Proposition}
\newtheorem{propnum}{Proposition}

\theoremstyle{plain}
\newtheorem{thm}{Theorem}

%-----------------Grabkiste nach Beweisen-------%
\renewcommand{\qedsymbol}{\(\hfill\blacksquare\)}
%
%
\usepackage[T1]{fontenc}
%\usepackage{tgschola}
\usepackage{stickstootext}
\usepackage[stickstoo,vvarbb]{newtxmath}
%
%
\usepackage{graphicx}
\usepackage{color}
%
%
\usepackage{extramarks}

\usepackage{hyperref}
\hypersetup{
	colorlinks=true,
	linktoc=all,
	linkcolor=blue,
	pdftoolbar=true,
	bookmarks=true,
	bookmarksnumbered=true
}

\usepackage{fancyhdr}
\setlength{\headheight}{15.2pt}
\renewcommand{\headrulewidth}{0pt}
\renewcommand{\footrulewidth}{0pt}
\pagestyle{fancy}
%
%
\usepackage{enumitem}
\setlist[enumerate, 1]{label=\thesection.\arabic*}
\setlist[enumerate, 2]{label=(\arabic*)}
\setlist[enumerate, 3]{label=\arabic*}

\newlist{rome}{enumerate}{1}
\setlist[rome]{label=\Roman*., align=right}

\newlist{alphbet}{enumerate}{1}
\setlist[alphbet]{label=\alph*.}

\begin{document}
\begin{titlepage}	
	\begin{center}
	
	\huge
	\textbf{Solutions Manual}
    
	\vspace{2em}
	\Large
	\textbf{A Book of Abstract Algebra - 2nd Edition}
    
	Charles C. Pinter
	
	\vspace{2em}
	\large
	\textbf{This solution manual was created by the MathLearners study group.}
	
	\end{center}
\end{titlepage}

\fancyhf{}
\cleardoublepage
\pagenumbering{Roman}
\setcounter{page}{3}
\tableofcontents

\cleardoublepage

\fancyhf{}
\pagenumbering{arabic}
\setcounter{page}{5}
\rhead{\thepage}
\lhead{\firstleftmark}
\sloppy

\setcounter{chapter}{1}
\chapter{Operations}
\section*{A: Examples of Operations}
\begin{enumerate}
    \item[3]
	\(a * b\) is a root of the equation \(x^2-a^2b^2 = 0\), on the set \(\mathbb{R}\).
	\begin{solution}
		From \(x^2-a^2b^2 = 0\) we get \(x^2=a^2b^2\), so \(\pm ab\) is a root, which means it's not unique. Thus \(a * b\) is not an operation on \(\mathbb{R}\).
	\end{solution}
	\item[5]
	Subtraction, on the set \(\{n \in \mathbb{Z} | n \geq 0\}\).
	\begin{solution}
		Subtraction is not an operation on that set, because if we have \(k, l \in \mathbb{Z}_{\geq0}\), where \(l > k\), we get a negative result, which would not be in \(\mathbb{Z}_{\geq0}\).
	\end{solution}
\end{enumerate}

\section*{B: Properties of Operations}
\begin{enumerate}
	\item[1] \(x * y = x + 2y + 4\)
	\begin{solution}
	We follow the steps from the example.
	\begin{rome}
		\item Commutative: \(x * y = x + 2y + 4\); \(y * x = y + 2x +4\). Thus \(x * y\) is not commutative.
		\item Associative: \(x * (y * z) = x*(y + 2z +4) = x + 2(y + 2z + 4) + 4\) \\ \((x * y) * z = (x + 2y + 4) * z = x + 2y + 4 + 2z\). Thus \(x * y\) is not associative.
		\item Solve \(x * e = x\) for \(e\).
		\begin{equation*}
		\begin{split}
			x * e & = x \\
			x + 2e + 4 & = x \\
			4 & = - 2e \\
			-2 & = e.
		\end{split}
		\end{equation*}
		\item Solve \(x * x' = e\) for \(x'\).
		\begin{equation*}
		\begin{split}
			x * x' & = e \\
			x + 2x' + 4 & = e \\
			x + 2x' + 4 & = -2 \\
			x + 2x' & = -6 \\
			2x' & = -6 - x \\
			x' & = -\frac{6+x}{2}.
		\end{split}
		\end{equation*}
	\end{rome}
	\end{solution}
	\item[2] \(x * y = x + 2y - xy\)
	\begin{solution}
	We follow the steps from the example.
	\begin{rome}
		\item Commutative:
		\begin{equation*}
		\begin{split}
			x * y = x + 2y - xy \\
			y * x = y + 2x - yx. \\
		\end{split}
		\end{equation*}
		Thus \(x * y\) is not commutative.
		
		\item Associative:
		\begin{equation*}
		\begin{split}
			x * (y * z) & = x * (y + 2z - yz) = x + 2(y + 2z - yz) - x(y + 2z - yz) \\
			(x * y) * z & = (x + 2y - xy) * z = x + 2y - xy + 2z - (x + 2y - xy)z.
		\end{split}
		\end{equation*}
		Thus \(x * y\) is not associative.
		
		\item Solve \(x * e = x\) for \(e\).
		\begin{equation*}
		\begin{split}
			x * e & = x \\
			x + 2e - xe & = x \\
			2e - xe & = 0 \\
			e(2 - x) & = 0 \\
			e & = 0.
		\end{split}
		\end{equation*}
		Check that it works: \(x * 0 = x + 2 \cdot 0 - x \cdot 0 = x + 0 - 0 = x\).
		
		\item Solve \(x * x' = e\) for \(x'\).
		\begin{equation*}
		\begin{split}
			x * x' & = e \\
			x + 2x' - xx' & = 0 \\
			x + x'(2 - x) & = 0 \\
			x'(2 - x) & = -x \\
			x' & = -\frac{x}{2 - x}.
		\end{split}
		\end{equation*}
		If \(x = 2\), then the right side is undefined. Thus there is no inverse.
	\end{rome}
	\end{solution}
	\item[3] \(x * y = |x + y|\)
	\begin{solution}
	We follow the steps from the example.
	\begin{rome}
		\item Commutative:
		\begin{equation*}
		\begin{split}
			x * y & = | x + y | \\
			y * x & = | y + x | = | x + y |.
		\end{split}
		\end{equation*}
		Thus \(x * y\) is commutative.
		
		\item Associative:
		\begin{equation*}
		\begin{split}
			x * (y * z) = x * | y + z | = \bigl| x + | y + z | \bigr| \\
			(x * y) * z = | x + y | * z = \bigl||x + y | + z \bigr|
		\end{split}
		\end{equation*}
		Let \(x = 2, y = -2\) and \(z = 0\). Then \(x * (y * z) = 2 * (-2 * 0) = \bigl| 2 + | -2 + 0 |\bigr| = | 2 + 2 | = 4.\) But \((x * y) * z = (2 * -2) * 0 = \bigl||2 + (-2)| + 0 \bigr| = |0 + 0 | = 0.\) Thus \(x * y\) is not associative.
		
		\item Solve \(x * e = x\) for \(e\).
		\begin{equation*}
		\begin{split}
			x * e & = x \\
			| x + e | & = x.
		\end{split}
		\end{equation*}
		But if \(x < 0\), then the right side is negative, but the left side is nonnegative, which is a contradiction. Thus there is no identity element.
	\end{rome}
	\end{solution}
\end{enumerate}


\chapter{The Definition of Groups}
\section*{C: Groups of Subsets of a Set}
\begin{enumerate}
	\item[1] Prove that there is an identity element with respect to the operation +, which is \(\emptysetAlt\).
	\begin{proof}
		Let \(A\) be any element of \(\mathcal{P}(D)\). Then \(A + \emptysetAlt = (A - \emptysetAlt) \cup (\emptysetAlt - A) = A \cup \emptysetAlt = A.\) Thus \(\emptysetAlt\) is the identity element.
	\end{proof}
	
	\item[2] Prove every subset \(A\) of \(D\) has an inverse with respect to +, which is \(A\).
	\begin{proof}
		Let \(A\) be any subset of \(D\). Then \(A + A = (A - A) \cup (A - A) = \emptysetAlt \cup \emptysetAlt = \emptysetAlt.\) Thus \(A\) is the inverse of \(A\).
	\end{proof}
\end{enumerate}

\section*{D: A Checkerboard Game}
\begin{enumerate}
	\item[1] Write the table of \(G\).
	\begin{solution}
		See table below: \\
		\begin{center}
		\begin{tabular}{c | c c c c}
			\(*\) & \(V\) & \(H\) & \(D\) & \(I\) \\
			\hline
			\(V\) & \(I\) & \(D\) & \(H\) & \(V\) \\
			\(H\) & \(D\) & \(I\) & \(V\) & \(H\) \\
			\(D\) & \(H\) & \(V\) & \(I\) & \(D\) \\
			\(I\) & \(V\) & \(H\) & \(D\) & \(I\)
		\end{tabular}
		\end{center}
	\end{solution}
	
	\item[2] Granting associativity, eplain why \(\langle G, *\rangle\) is a group.
	\begin{solution}
		Assuming associativity, we only have to show that there exists an identity element and an inverse.
		
		\begin{propnum}
			The identity element of \(G\) is \(I\).
		\end{propnum}
		\begin{proof}
			Let \(X\) be any element of \(G\). Then \(X * I = X\) and \(I * X = X\). Thus \(I\) is the identity element of \(G\).
		\end{proof}
		
		\begin{propnum}
			For any element \(X\) of \(G\), the inverse is \(X\).
		\end{propnum}
		\begin{proof}
			Let \(X\) be any element of \(G\). Obersve that \(X * X = I\) and \(X * X = I\). Thus \(X\) is the inverse of \(X\).
		\end{proof}
	\end{solution}
	
	We also note that \(G\) is an abelian group, since changing the order of the operands doesn't change the result (i.e. \(D * V = V * D = H\)).
\end{enumerate}

\section*{E: A Coin Game}
\begin{enumerate}
	\item[1] If \(G = \{I, M_1, ..., M_7\}\) and \(*\) is the operation we have just defined, write the table of \(\langle G, * \rangle\).
	\begin{solution}
		See table below: \\
		\begin{center}
		\begin{tabular}{ c | c c c c c c c c }
			\(I\) & \(I\) & \(M_1\) & \(M_2\) & \(M_3\) & \(M_4\) & \(M_5\) & \(M_6\) & \(M_7\) \\
			\hline
			\(I\) & \(I\) & \(M_1\) & \(M_2\) & \(M_3\) & \(M_4\) & \(M_5\) & \(M_6\) & \(M_7\) \\
			\(M_1\) & \(M_1\) & \(I\) & \(M_3\) & \(M_2\) & \(M_5\) & \(M_4\) & \(M_7\) & \(M_6\) \\
			\(M_2\) & \(M_2\) & \(M_3\) & \(I\) & \(M_1\) & \(M_6\) & \(M_7\) & \(M_4\) & \(M_5\) \\
			\(M_3\) & \(M_3\) & \(M_2\) & \(M_1\) & \(I\) & \(M_7\) & \(M_6\) & \(M_5\) & \(M_4\) \\
			\(M_4\) & \(M_4\) & \(M_6\) & \(M_5\) & \(M_7\) & \(I\) & \(M_2\) & \(M_1\) & \(M_3\) \\
			\(M_5\) & \(M_5\) & \(M_7\) & \(M_4\) & \(M_6\) & \(M_1\) & \(M_3\) & \(I\) & \(M_2\) \\
			\(M_6\) & \(M_6\) & \(M_4\) & \(M_7\) & \(M_5\) & \(M_2\) & \(I\) & \(M_3\) & \(M_1\) \\
			\(M_7\) & \(M_7\) & \(M_5\) & \(M_6\) & \(M_4\) & \(M_3\) & \(M_1\) & \(M_2\) & \(I\)
		\end{tabular}
		\end{center}
	\end{solution}
	
	\item[2] Granting associativity, eplain why \(\langle G, * \rangle\) is a group. Is it commutative? If not, show why not.
	\begin{solution}
		As can be seen from the operation table, there exists an identity element, namely \(I\), and every element has an inverse. But \(G\) is not abelianc, since \(M_4 * M_5 = M_2\), but \(M_5 * M_4 = M_1\).
	\end{solution}
\end{enumerate}

\section*{F: Groups in Binary Codes}
\begin{enumerate}
	\item[1] Show that \((a_1, a_2, ..., a_n) + (b_1, b_2, ..., b_n) = (b_1, b_2, ..., b_n) + (a_1, a_2, ..., a_n).\)
	\begin{proof}
		We use induction and start with the base case:
		\begin{rome}
			\item \(0 + 1 = 1 = 1 + 0\), and \\
			\(0 + 0 = 0 = 1 + 1.\)
			\item Assume that \((a_1, a_2, ..., a_k) + (b_1, b_2, ..., b_k) = (b_1, b_2, ..., b_k) + (a_1, a_2, ..., a_k)\) for \(k \leq n\). Observe that \((a_1, a_2, ..., a_k, a_{k+1}) + (b_1, b_2, ..., b_k, b_{k+1}) = (b_1 + a_1, b_2 + a_2, ..., b_k + a_k, a_{k+1} + b_{k+1})\). Thus we have to show that \(a_{k+1} + b_{k+1} = b_{k+1} + a_{k+1}\). Since \(a_{k+1}\) can be either 0 or 1, and \(b_{k+1}\) can also be either 0 or 1, we refer to the base case to conclude that \(a_{k+1} + b_{k+1} = b_{k+1} + a_{k+1}\). Thus we have \((a_1, a_2, ..., a_k, a_{k+1}) + (b_1, b_2, ..., b_k, b_{k+1}) = (b_1, b_2, ..., b_k, b_{k+1}) + (a_1, a_2, ..., a_k, a_{k+1})\).
		\end{rome}
		This completes our proof that \((a_1, a_2, ..., a_n) + (b_1, b_2, ..., b_n) = (b_1, b_2, ..., b_n) + (a_1, a_2, ..., a_n)\).
	\end{proof}
	
	\item[2] Check the remaining six cases:
	\begin{solution}
	\begin{align*}
		1 + (0 + 1) = 1 + 1 = 0 = 1 + 1 = (1 + 0) + 1 \\
		1 + (0 + 0) = 1 + 0 = 1 = 1 + 0 = (1 + 0) + 0 \\
		0 + (1 + 1) = 0 + 0 = 0 = 1 + 1 = (0 + 1) + 1 \\
		0 + (1 + 0) = 0 + 1 = 1 = 1 + 0 = (0 + 1) + 0 \\
		0 + (0 + 1) = 0 + 1 = 1 = 0 + 1 = (0 + 0) + 1 \\
		0 + (0 + 0) = 0 + 0 = 0 = 0 + 0 = (0 + 0) + 0
	\end{align*}
	\end{solution}
	
	\item[3] Show that \((a_1, ..., a_n) + \left[(b_1, ..., b_n) + (c_1, ..., c_n)\right] = \left[(a_1, ..., a_n) + (b_1, ..., b_n)\right] + (c_1, ..., c_n)\).
	\begin{proof}
		We use proof by induction and start with the base case.
		\begin{rome}
			\item See exercise 2.
			\item Assume \((a_1, ..., a_k) + \left[(b_1, ..., b_k) + (c_1, ..., c_k)\right] = \left[(a_1, ..., a_k) + (b_1, ..., b_k)\right] + (c_1, ..., c_k)\) for \(k \leq n\). Observe that the first \(k\) digits in \((a_1, ..., a_k, a_{k+1}) + \left[(b_1, ..., b_k, b_{k+1}) + (c_1, ..., c_k, c_{k+1})\right]\) are associative, which means whe only have to show that \(a_{k+1} + (b_{k+1} + c_{k+1}) = (a_{k+1} + b_{k+1}) + c_{k+1}\) is true. And since we have shown in the base case that any binary word of length 1 is associative, we conclude that \(a_{k+1} + (b_{k+1} + c_{k+1}) = (a_{k+1} + b_{k+1}) + c_{k+1}\) holds, and thus \((a_1, ..., a_k, a_{k+1}) + \left[(b_1, ..., b_k, b_{k+1}) + (c_1, ..., c_k, c_{k+1})\right] = \left[(a_1, ..., a_k, a_{k+1}) + (b_1, ..., b_k, b_{k+1})\right] + (c_1, ..., c_k, c_{k+1})\).
		\end{rome}
		This completes our proof that addition of binary words is associative.
	\end{proof}
	
	\item[6] Show that \(A + B = A - B\), [where \(A - B = A + (-B)\)].
	\begin{proof}
		Observe that:
		\begin{equation*}
		\begin{split}
			A & = A \\
			& = A + \mathbb{0} \\
			& = A + (B + B) \\
			& = (A + B) + B \\
			A + (-B) & = (A + B) + B + (-B) \\
			& = (A + B) + (B - B) \\
			& = (A + B) + \mathbb{0} \\
			A - B & = A + B.
		\end{split}
		\end{equation*}
		This completes our proof that \(A + B = A - B\).
	\end{proof}
	
	\item[7] If \(A + B = C\), show that \(A = B + C\).
	\begin{proof}
		Observe that:
		\begin{equation*}
		\begin{split}
			A + B & = C \\
			(A + B) + B & = C + B \\
			A + (B + B) & = \\
			A + \mathbb{0} & = \\
			A & = B + C.
		\end{split}
		\end{equation*}
		This completes our proof that \(A + B = C\) implies \(A = B + C\).
	\end{proof}
\end{enumerate}
\end{document}