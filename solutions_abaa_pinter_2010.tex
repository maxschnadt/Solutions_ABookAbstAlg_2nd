\documentclass[11pt, b5paper, draft, fleqn]{book}
\usepackage[inner=2.00cm, outer=2.0cm, top=2.0cm, bottom=2.0cm]{geometry}
%
%
\usepackage[english]{babel}
%
%
\usepackage{amsmath}
\usepackage{amsfonts}
%\usepackage{amssymb}
\usepackage{amsthm}

\theoremstyle{remark}
\newtheorem*{solution}{Solution}

%-----------------Grabkiste nach Beweisen-------%
\renewcommand{\qedsymbol}{\(\hfill\blacksquare\)}
%
%
\usepackage[T1]{fontenc}
%\usepackage{tgschola}
\usepackage{stickstootext}
\usepackage[stickstoo,vvarbb]{newtxmath}
%
%
\usepackage{graphicx}
\usepackage{color}
%
%
\usepackage{extramarks}

\usepackage{hyperref}
\hypersetup{
	colorlinks=true,
	linktoc=all,
	linkcolor=blue,
	pdftoolbar=true,
	bookmarks=true,
	bookmarksnumbered=true
}

\usepackage{fancyhdr}
\setlength{\headheight}{15.2pt}
\renewcommand{\headrulewidth}{0pt}
\renewcommand{\footrulewidth}{0pt}
\pagestyle{fancy}
%
%
\usepackage{enumitem}
\setlist[enumerate, 1]{label=\thesection.\arabic*}
\setlist[enumerate, 2]{label=(\arabic*)}
\setlist[enumerate, 3]{label=\arabic*}

\newlist{rome}{enumerate}{1}
\setlist[rome]{label=\Roman*., align=right}

\newlist{alphbet}{enumerate}{1}
\setlist[alphbet]{label=\alph*.}

\begin{document}
\begin{titlepage}	
	\begin{center}
	
	\huge
	\textbf{Solutions Manual}
    
	\vspace{2em}
	\Large
	\textbf{A Book of Abstract Algebra - 2nd Edition}
    
	Charles C. Pinter
	
	\vspace{2em}
	\large
	\textbf{This solution manual was created by the MathLearners study group.}
	
	\end{center}
\end{titlepage}

\fancyhf{}
\cleardoublepage
\pagenumbering{Roman}
\setcounter{page}{3}
\tableofcontents

\cleardoublepage

\fancyhf{}
\pagenumbering{arabic}
\setcounter{page}{5}
\rhead{\thepage}
\lhead{\firstleftmark}

\chapter{Operations}
\section{A: Examples of Operations}
\begin{enumerate}
    \item[3]
	\(a * b\) is a root of the equation \(x^2-a^2b^2 = 0\), on the set \(\mathbb{R}\).
	\begin{solution}
		From \(x^2-a^2b^2 = 0\) we get \(x^2=a^2b^2\), so \(\pm ab\) is a root, which means it's not unique. Thus \(a * b\) is not an operation on \(\mathbb{R}\).
	\end{solution}
	\item[5]
	Subtraction, on the set \(\{n \in \mathbb{Z} | n \geq 0\}\).
	\begin{solution}
		Subtraction is not an operation on that set, because if we have \(k, l \in \mathbb{Z}_{\geq0}\), where \(l > k\), we get a negative result, which would not be in \(\mathbb{Z}_{\geq0}\).
	\end{solution}
\end{enumerate}
\section{B: Properties of Operations}
\begin{enumerate}
	\item[1] \(x * y = x + 2y + 4\)
	\begin{solution}
	We follow the steps from the example.
	\begin{rome}
		\item Commutative: \(x * y = x + 2y + 4\); \(y * x = y + 2x +4\). Thus \(x * y\) is not commutative.
		\item Associative: \(x * (y * z) = x*(y + 2z +4) = x + 2(y + 2z + 4) + 4\) \\ \((x * y) * z = (x + 2y + 4) * z = x + 2y + 4 + 2z\). Thus \(x * y\) is not associative.
		\item Solve \(x * e = x\) for \(e\).
		\begin{equation*}
		\begin{split}
			x * e & = x \\
			x + 2e + 4 & = x \\
			4 & = - 2e \\
			-2 & = e.
		\end{split}
		\end{equation*}
		\item Solve \(x * x' = e\) for \(x'\).
		\begin{equation*}
		\begin{split}
			x * x' & = e \\
			x + 2x' + 4 & = e \\
			x + 2x' + 4 & = -2 \\
			x + 2x' & = -6 \\
			2x' & = -6 - x \\
			x' & = -\frac{6+x}{2}.
		\end{split}
		\end{equation*}
	\end{rome}
	\end{solution}
	\item[2] \(x * y = x + 2y - xy\)
	\begin{solution}
	We follow the steps from the example.
	\begin{rome}
		\item Commutative:
		\begin{equation*}
		\begin{split}
			x * y = x + 2y - xy \\
			y * x = y + 2x - yx. \\
		\end{split}
		\end{equation*}
		Thus \(x * y\) is not commutative.
		
		\item Associative:
		\begin{equation*}
		\begin{split}
			x * (y * z) & = x * (y + 2z - yz) = x + 2(y + 2z - yz) - x(y + 2z - yz) \\
			(x * y) * z & = (x + 2y - xy) * z = x + 2y - xy + 2z - (x + 2y - xy)z.
		\end{split}
		\end{equation*}
		Thus \(x * y\) is not associative.
		
		\item Solve \(x * e = x\) for \(e\).
		\begin{equation*}
		\begin{split}
			x * e & = x \\
			x + 2e - xe & = x \\
			2e - xe & = 0 \\
			e(2 - x) & = 0 \\
			e & = 0.
		\end{split}
		\end{equation*}
		Check that it works: \(x * 0 = x + 2 \cdot 0 - x \cdot 0 = x + 0 - 0 = x\).
		
		\item Solve \(x * x' = e\) for \(x'\).
		\begin{equation*}
		\begin{split}
			x * x' & = e \\
			x + 2x' - xx' & = 0 \\
			x + x'(2 - x) & = 0 \\
			x'(2 - x) & = -x \\
			x' & = -\frac{x}{2 - x}.
		\end{split}
		\end{equation*}
		If \(x = 2\), then the right side is undefined. Thus there is no inverse.
	\end{rome}
	\end{solution}
	\item[3] \(x * y = |x + y|\)
	\begin{solution}
	We follow the steps from the example.
	\begin{rome}
		\item Commutative:
		\begin{equation*}
		\begin{split}
			x * y & = | x + y | \\
			y * x & = | y + x | = | x + y |.
		\end{split}
		\end{equation*}
		Thus \(x * y\) is commutative.
		
		\item Associative:
		\begin{equation*}
		\begin{split}
			x * (y * z) = x * | y + z | = \bigl| x + | y + z | \bigr| \\
			(x * y) * z = | x + y | * z = \bigl||x + y | + z \bigr|
		\end{split}
		\end{equation*}
		Let \(x = 2, y = -2\) and \(z = 0\). Then \(x * (y * z) = 2 * (-2 * 0) = \bigl| 2 + | -2 + 0 |\bigr| = | 2 + 2 | = 4.\) But \((x * y) * z = (2 * -2) * 0 = \bigl||2 + (-2)| + 0 \bigr| = |0 + 0 | = 0.\) Thus \(x * y\) is not associative.
		
		\item Solve \(x * e = x\) for \(e\).
		\begin{equation*}
		\begin{split}
			x * e & = x \\
			| x + e | & = x.
		\end{split}
		\end{equation*}
		But if \(x < 0\), then the right side is negative, but the left side is nonnegative, which is a contradiction. Thus there is no identity element.
	\end{rome}
	\end{solution}
\end{enumerate}
\end{document}